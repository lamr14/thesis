% !TEX root = ../TUCthesis.tex
%************************************************
\chapter{Introduction}\label{ch:introduction}
%************************************************

\bigskip
\section{Motivation}
\bigskip
\section{Previous Work}
\bigskip

(OF el protocolo del futuro){
For this reason, universities and research groups are exploring new techniques to deal and to make easier to handle a LAN or WAN administration.

Many hardware vendors are paying their attention to \ac{SDN}; hence the interest to know about this technology.
SDN defines a new relationship between the network devices  and the software that controls them.
There is the possibility of control the whole network from a single point.

Since the launch of SDN, there has been some interesting projects/applications: Network virtualization (Flowvisor), load balancing (PlugNServe), traffic engineering (Aggregation) amont others.

Changes in the network just require the administrator to update of the links in the controller.

NOX is one of the first controllers used to make tests in SDN networks and it achieved a good performance.

Depending on the network size, it is possible to apply a distributed controller (various controllers) or centralized (one controller). (insertar figura control distribuido)

OpenFlow "Permite acceder directamente y manipular el plano de redireccionamiento de dispositivos de red como switches y routers, ya sean físicos o virtuales (HP 2008)"
Facilita el acceso a dispositivos de red mediante una interfaz standard. (Figura: arquitectura de comunicación OpenFlow, Heller 2011)
OpenFlow utiliza TCP/SSL (capas 4 y 5 del modelo OSI) para la comunicación del plano de control y el controlador.
OpenFlow es un protocolo para operar redes SDN. Al igual que TCP, su estructura está diseñada por mensajes que establecen una comunicación y generan las acciones correspondientes(McKeown et al, 2008).
OpenFlow es el lenguaje de comunicación entre en controlador y los switches (southbound interface).

The last version of the protocol is...
}

By separating the data plane from the control plane, it is possible to have a better control of the network, thus heading to more efficiency.
In traditional networks: proprietary devices, hard to integrate, standards for interoperability were missing (figura redes clásicas vs. OpenFlow arch, Sherwood et al 2009)
Al desacoplar el control de datos, est significa que esto puede delegarse a un controlador externo, o sea, se puede programar desde fuera del dispositivo la manera en que serán procesados los flujos de paquetes.
Así es posible conocer que orígenes y destinos conoce el dispositivo.

\section{Research Questions}

\begin{itemize}
  \item Analyze the performance of the controllers in a simulated environment, finding the possible limitations of each controller.
  \item Research and develop a proof of concept for SDN controllers under different administration domains, where the APIs of each controller have to communicate with each other.
  \item Run tests of performance and throughput, in order to check the interoperability of different solutions interconnected.
\end{itemize}

\bigskip
\section{Outline}
The organization of this work is structured as follows:

\autoref{ch:relwork} looks over some of the related research done in the field.
\autoref{ch:relwork}
\autoref{ch:methodology}
\autoref{ch:evaluation}
\autoref{ch:results}
\autoref{ch:conclusion} the conclusions gained and possible future work

 \ac{SDN}

%The results are presented and explained in \autoref{ch:results}.
%General conclusions and findings about the project will be discussed in \autoref{ch:conclusion}.

Mininet 1\cite{Mininet}

Mininet 2\cite{team2014mininet}

\cite{feamster2013road}

\cite{rowshanrad2016performance}

\cite{gomez2013openflow}

ODL \cite{odl}

FL \cite{fl}

FL 2 \cite{projectfloodlight}

